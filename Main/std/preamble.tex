%!TEX root = /Users/dbreuer/Documents/Work/_FH/_Master/master_thesis/Main/Master Thesis.tex

%%% PREAMBLE

\documentclass[12pt,
               headsepline,
               DIV14, % Seitenränder festlegen
               BCOR5mm, % Bindekorrektur
               a4paper,
               oneside,
               cleardoublestandard,
               openany,
               bibtotoc,
               liststotoc,
               halfparskip,
               pointlessnumbers,
               final
               ]{scrbook}

% Paket für Zeilenabstände
\usepackage{setspace}
\raggedbottom

\usepackage{scrpage2}
\pagestyle{scrheadings}
\clearscrheadfoot
\cfoot{\pagemark}
% \ihead{\headmark}
\ohead{\headmark}
% \automark{section}

% include XMP-Metadata
\usepackage{xmpincl} 
\includexmp{std/metadata}

% Wie viele Ebenen im Inhaltsverzeichnis?
\setcounter{tocdepth}{3}

% Bis zu welcher Ebene sollen die Abschnitt nummeriert werden
\setcounter{secnumdepth}{3}

% Grafiken und Farbe
\usepackage[pdftex]{graphicx} % for graphic handling
\usepackage[usenames,dvipsnames,pdftex]{color} % for color handling
% Einige Farben
\definecolor{lightgray}{gray}{0.9}
\definecolor{cmd_line}{rgb}{0.4,.8,1}
\definecolor{comment}{rgb}{0.2,0.7,0.5}

% Paket für Rotation
\usepackage{rotating}

% Paket für Listings
\usepackage{listings}
\usepackage{dvsm} % Use DejaVu Sans Mono as Monospace Font
% get closed frames on each page for listings
\usepackage{framed}
\newenvironment{kasten}{%
  \def\FrameCommand{\fboxsep=\FrameSep \colorbox{lightgray}}%
  \MakeFramed {\hsize=0.9\textwidth \FrameRestore}}%
 {\endMakeFramed}

\lstloadlanguages{Java,XML,Ruby,HTML}
\newcommand{\listingname}{Listing}
\lstset{
  numbers=left, 
	numberstyle=\ttfamily\tiny, 
	stepnumber=1,
	lineskip=2pt,
	numbersep=5pt,
	language=Java,
	breaklines=true,
	breakautoindent=true,
	postbreak=\space,
	tabsize=2,
	frame=single,
  keywordstyle=\bfseries\color{blue},
  stringstyle=\color{ForestGreen},
	numberfirstline=false,
  commentstyle=\slshape\scriptsize\color{Gray},
  basicstyle=\ttfamily\scriptsize,
  % morekeywords={while,condition,invoke,receive,assign,pipe},
  keywordstyle=[2]\ttfamily\redhighlight,
  keywordstyle=[3]\ttfamily\yellowhighlight,
  keywordstyle=[4]\ttfamily\slshape
}

% bg colored text for highlighting

\newcommand{\redhighlight}[1]{\colorbox{red}{\textcolor{white}{#1}}}
\newcommand{\yellowhighlight}[1]{\colorbox{yellow}{\textcolor{black}{#1}}}

% Listings can be included as following:
% 
% include a file (from line n to m): \lstinputlisting[firstline=n,lastline=m]
% 
% standard listing: \begin{lstlisting} ... \end{lstlisting}
% 
% Inline: \lstinline+code here+

\lstdefinelanguage{YAML}{
  morekeywords = {uri, successors, type, output, namespace, description, predecessors, input},
  % moredelim=[is][\color{green}]{"},
  emph={[2]null},
  emphstyle=[2]\bfseries\color{red},
  moredelim=[l][\color{OliveGreen}]{"},
  morecomment=[l][commentstyle]{\#},
}
% 
% \lstdefinelanguage{Turtle}{
%     morekeywords = {@prefix},
%   emphstyle=\itshape %\underbar
%   % emph={[2]mit,sonst},
%   % emphstyle=[2]\color{red},
%   % moredelim=[is][\color{green}]{/*}{*/}
% }

% Use utf-8 encoding for foreign characters
\usepackage[utf8]{inputenc}

% Paket das die Ausgabefonts definiert
% \usepackage[T1]{fontenc}

% Paket um jede Schriftgröße zuzulassen
% \usepackage{type1cm}

% This is now the recommended way for checking for PDFLaTeX:
\usepackage{ifpdf}

% change enumeration package
\usepackage{enumerate}

% Package for verbatim text
% \usepackage{verbatim}
% Notwendiges Paket, um Verbatim in Fußzeilen zu setzen
% \usepackage{fancyvrb}
% \VerbatimFootnotes

% schoenere Kennzeichnung im Literaturverzeichnis
\usepackage[square,sort]{natbib}
% Festlegung Art der Zitierung - Havardmethode: Abkuerzung Autor + Jahr %
% BibTeX Style nach Norm DIN 1505 %
\bibliographystyle{natdin}

% Schriftstile umsetzen
% \setkomafont{sectioning}{\normalfont\normalcolor\bfseries}
\setkomafont{descriptionlabel}{\normalfont\normalcolor\bfseries}
\setkomafont{captionlabel}{\usekomafont{descriptionlabel}}
\setkomafont{dictumtext}{\normalfont\normalcolor\itshape}
\setkomafont{dictumauthor}{\scshape}

% Hurenkinder und Schusterjungen verhindern
\clubpenalty = 10000
\widowpenalty = 10000
\displaywidowpenalty = 10000

% Paket für die Verwendung von URLs durch den Befehl \url{}
\usepackage{url}

%\usepackage{array}
\usepackage{colortbl}

% Babelpaket für deutsche Bezeichner
\usepackage[ngerman]{babel}

% For syntax Checking
\usepackage{syntonly}
% \syntaxonly % Comment out for Output

\usepackage{makeidx}
% \makeindex

% Mathepakete
\usepackage{amssymb} %maths
\usepackage{amsmath} %maths
\usepackage{amsthm}

% Mathematische Definition
\newtheorem{definition}{Definition}
% \usepackage{mathabx}

% Springt im PDFViewer an die Stelle an der gerade editiert wurde
\usepackage{pdfsync}

% Paket und Einstellungen für das Abkürzungsverzeichnis
\usepackage[norefpage,noprefix,german,intoc]{nomencl}
% Befehl umbenennen in abk
\let\abk\nomenclature
% Deutsche Überschrift
\renewcommand{\nomname}{Abkürzungsverzeichnis}
% Punkte zw. Abkürzung und Erklärung
\renewcommand{\nomlabel}[1]{#1 \dotfill}
% Zeilenabstände verkleinern
\setlength{\nomitemsep}{-\parsep}
% Definiert die Aufteilung im Glossar zwischen Begriffen und Erläuterung
\setlength{\nomlabelwidth}{.25\hsize}
%\makeglossary
\makenomenclature

%Paket für ein deutsches Literaturverzeichnis
\usepackage{bibgerm}

% other commands
\newcommand{\note}[1]{\textbf{#1}}

\pdfpagewidth=\paperwidth \pdfpageheight=\paperheight
\usepackage[pdftex,plainpages=false,pdfpagelabels,
            pdftitle={Konzeption, prototypische Realisierung und szenariobasierte Validierung einer dienstorientierten Multimediaarchitektur},
            pdfauthor={Dirk Breuer - University of Applied Science, Cologne},
            pdfkeywords={Architektur,SOA,Prototyping,COSIMA,Validierung}]{hyperref} % Has to stand at the end of the preamble

%%% END PREAMBLE
