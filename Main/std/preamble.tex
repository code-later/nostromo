%!TEX root = /Users/dbreuer/Documents/Work/_FH/_Master/master_thesis/Main/Master Thesis.tex

%%% PREAMBLE

\documentclass[12pt,headsepline,DIV12,BCOR12mm,a4paper,oneside,cleardoublestandard,openany,bibtotoc,liststotoc,halfparskip]{scrbook}

% package for typearea settings
%\usepackage[DIV15]{typearea}

% package declaration
\usepackage{setspace}
% \onehalfspacing
\raggedbottom

%package for inserting source code
\usepackage{listings}
\lstloadlanguages{Java,XML,Ruby,HTML}
\newcommand{\listingname}{Listing}
\lstset{numbers=left, 
	numberstyle=\tiny, 
	stepnumber=1,
	lineskip=3pt,
	numbersep=5pt,
	language=Java,
	breaklines=true,
	breakautoindent=true,
	postbreak=\space,
	tabsize=3,
	frame=single,
  keywordstyle=\bfseries,
	numberfirstline=false,
  commentstyle=\ttfamily\footnotesize,
  basicstyle=\ttfamily\footnotesize,
	morekeywords={sequence,while,condition,invoke,receive,assign,pipe}}

\raggedbottom
%\typearea[current]{calc}

% Use utf-8 encoding for foreign characters
\usepackage[utf8]{inputenc}

% Paket das die Ausgabefonts definiert
% \usepackage[T1]{fontenc}

% This is now the recommended way for checking for PDFLaTeX:
\usepackage{ifpdf}

% Grafiken und Farbe
\usepackage[pdftex]{graphicx} % for graphic handling
\usepackage[pdftex]{color} % for color handling

\usepackage{verbatim}

% schoenere Kennzeichnung im Literaturverzeichnis
\usepackage[square]{natbib}
% Festlegung Art der Zitierung - Havardmethode: Abkuerzung Autor + Jahr %
% BibTeX Style nach Norm DIN 1505 %
\bibliographystyle{natdin}

% Hurenkinder und Schusterjungen verhindern
\clubpenalty = 10000
\widowpenalty = 10000
\displaywidowpenalty = 10000

%Paket für die Verwendung von URLs durch den Befehl \url{}
\usepackage{url}

\usepackage{fancyvrb}

%\usepackage{array}
\usepackage{colortbl}
\usepackage[ngerman]{babel}
\usepackage{syntonly}
\usepackage{makeidx}
\usepackage{amssymb} %maths
\usepackage{amsmath} %maths
\usepackage{amsthm}
\usepackage{vmargin}
% \usepackage{mathabx}
% \makeindex

% Paket und Einstellungen für das Abkürzungsverzeichnis
\usepackage[refpage,noprefix,german,intoc]{nomencl}
% Befehl umbenennen in abk
\let\abk\nomenclature
% Deutsche Überschrift
\renewcommand{\nomname}{Abkürzungsverzeichnis}
% Punkte zw. Abkürzung und Erklärung
\renewcommand{\nomlabel}[1]{#1 \dotfill}
% Zeilenabstände verkleinern
\setlength{\nomitemsep}{-\parsep}
% Definiert die Aufteilung im Glossar zwischen Begriffen und Erläuterung
\setlength{\nomlabelwidth}{.25\hsize}
%\makeglossary
\makenomenclature

%Paket für ein deutsches Literaturverzeichnis
\usepackage{bibgerm}

\VerbatimFootnotes

% other commands
\newcommand{\note}[1]{\textbf{#1}}
% \newcommand{\url}[1]{\emph{#1}}
\pdfpagewidth=\paperwidth \pdfpageheight=\paperheight
% \syntaxonly % Comment out for Output
\definecolor{lightgrey}{gray}{0.95}
\definecolor{cmd_line}{rgb}{0.4,.8,1}
\definecolor{comment}{rgb}{0.2,0.7,0.5}
\usepackage[pdftex,plainpages=false,pdfpagelabels,
            pdftitle={Evaluation der MIAV Architektur anhand eines Anwendungszenarios durch prototypische Implementierung},
            pdfauthor={Dirk Breuer - University of Applied Science, Cologne},
            pdfkeywords={}]{hyperref} % Has to stand at the end of the preamble

%%% END PREAMBLE