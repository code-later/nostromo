%!TEX root = /Users/dbreuer/Documents/Work/_FH/_Master/master_thesis/Main/Master Thesis.tex

\chapter{Prototypische Realisierung} % (fold)
\label{cha:prototypische_realisierung}

- Vorgehen bei der Realisierung erläutern
- Kurz (!!) auf das Santiago Projekt eingehen (da es im allerersten Schritt dazu gedient hat, eine erste funktionierende Grundlage der Architektur zu schaffen)
- Wichtige Punkte herausarbeiten
- Auf Implementierungsdetails nur an grundlegenden Stellen eingehen
- Schwierigkeiten und vor allem deren Problemlösungen darstellen
- Auswirkungen dieser Probleme/Lösungen für die Architektur
- Fokus vor allem im Text auf das Vorgehen und Wendepunkte
- Hauptteil der Quellcode
- Funktionierenden Code mit Build Tool und Dokumentation ausliefern (!!)

\section{Realisierung der Architektur} % (fold)
\label{sec:realisierung_der_architektur}

  - Santiago Projekt
  - bottom-up Ansatz
  
  
  ZUR SERVICEKOMPOSITION: Aus den in~\ref{sub:service_komposition} genannten Gründen kann auch bei der prototypische Realisierung auf die Verwendung einer existierenden Prozessbeschreibungssprache verzichtet werden.
  
\subsection{Vorgehen} % (fold)
\label{sub:vorgehen_architektur}

  - wie wurde bei der Umsetzung der Architektur vorgegangen?

% subsection vorgehen_architektur (end)
  
\subsection{Probleme und Lösungen} % (fold)
\label{sub:probleme_und_loesungen_architektur}

  - Welche Probleme sind aufgetaucht?
  - Wie wurden diese gelöst?
  - Auswirkungen auf die Architektur!
  - Workflow ohne externes Messaging System möglich!?

% subsection probleme_und_loesungen_architektur (end)

% section realisierung_der_architektur (end)

\section{Realisierung des Szenario} % (fold)
\label{sec:realisierung_des_szenario}

  - Nerstrand Projekt
  - wichtige Punkte beschreiben
  - top-down Ansatz

\subsection{Extrahierte Anforderungen} % (fold)
\label{sub:extrahierte_anforderungen}

% subsection extrahierte_anforderungen (end)

\subsection{Vorgehen} % (fold)
\label{sub:vorgehen_szenario}

  - wie wurde bei der Umsetzung des Anwendungsszenario vorgegangen?

% subsection vorgehen_szenario (end)

\subsection{Probleme und Lösungen} % (fold)
\label{sub:probleme_und_loesungen_szenario}

  - Welche Probleme sind aufgetaucht?
  - Wie wurden diese gelöst?
  - Auswirkungen auf die Architektur!

% subsection probleme_und_loesungen_szenario (end)

% section realisierung_des_szenario (end)

% chapter prototypische_realisierung (end)