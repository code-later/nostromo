\chapter{Prototypische Realisierung} % (fold)
\label{cha:prototypische_realisierung}

- Vorgehen bei der Realisierung erläutern
- Kurz (!!) auf das Santiago Projekt eingehen (da es im allerersten Schritt dazu gedient hat, eine erste funktionierende Grundlage der Architektur zu schaffen)
- Wichtige Punkte herausarbeiten
- Auf Implementierungsdetails nur an grundlegenden Stellen eingehen
- Schwierigkeiten und vor allem deren Problemlösungen darstellen
- Auswirkungen dieser Probleme/Lösungen für die Architektur
- Fokus vor allem im Text auf das Vorgehen und Wendepunkte
- Hauptteil der Quellcode
- Funktionierenden Code mit Build Tool und Dokumentation ausliefern (!!)

% chapter prototypische_realisierung (end)