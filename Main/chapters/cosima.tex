%!TEX root = /Users/dbreuer/Documents/Work/_FH/_Master/master_thesis/Main/Master Thesis.tex

\chapter{COSIMA - Eine dienstorientierten Multimediaarchitektur} % (fold)
\label{cha:eine_dienstorientierten_multimediaarchitektur}

  Im Abschnitt~\ref{sec:motivation} wurde bereits kurz darauf eingegangen, dass die vorliegenden Arbeit vor dem Hintergrund des COSIMA-Projekts entstanden ist. In diesem Kapitel soll dieses Projekt so weit vorgestellt werden, dass für den weiteren Verlauf der Arbeit ein grundlegendes Verständnis über die Ziele, Alleinstellungsmerkmale und Herausforderungen existiert.
  
\section{Motivation und Ziele von COSIMA} % (fold)
\label{sec:motivation_und_ziele_von_cosima}

  Das COSIMA-Projekt ist aus dem Wahlpflichtfach Modellierung in audio-visuellen Medien (MIAV\abk{MIAV}{Modellierung in audio-visuellen Medien}) an der Fachhochschule Köln im Masterstudiengang der Medieninformatik hervorgegangen. Im Rahmen einer Projektarbeit wurde das die Projektidee weiter ausgearbeitet und konzeptioniert. Die Ergebnisse dieser Arbeit wurden als Institutsbericht an Fachhochschule Köln bereitsgestellt und sind dort im Detail einsehbar~\citep{bericht}.
  
  Das folgende Kapitel wird daher nur auf die wesentliche Punkte des COSIMA-Projekts eingehen und ihre Relevanz für diese Arbeit herausstellen. Die ursprüngliche Idee hinter COSIMA war es ein Framework zu entwickeln, dass die Entwicklung von Multimediaanwendungen vereinfacht. Im Gegensatz zu anderen Medienframeworks, wie etwa dem \emph{Java Media Framework} (JMF\abk{JMF}{Java Media Framework}), wurde bei dem COSIMA-Projekt ein ganzheitlicher Ansatz verfolgt.
  
  Im Institutsbericht wird darauf hingewiesen, "`dass die Entwicklung von Multimediaanwendungen derzeit verhältnismässig aufwendig ist"'~\citep[S. 2]{bericht}. Eine Ursache dieser Problematik, liegt nach Aussage der Autoren darin begründet, dass sich die zur Zeit verfügbaren Rahmenwerke im Bereich der Multimediaverarbeitung auf einen sehr engen Einsatzbereich beschränken. Neben JMF sind hier zusätzlich noch \emph{QuickTime}\footnote{\url{http://www.apple.com/quicktime/}} und \emph{ImageJ}\footnote{\url{http://rsbweb.nih.gov/ij/}} zu nennen. Andere Aspekte von Multimediaanwendungen, wie etwa die Integration von Metadaten, müssten von dem Anwendungsentwickler erst manuell mit diesen Rahmenwerken integriert werden. "`Ein Meta-Framework, welches die bestehenden Ansätze verbinden und integrieren könnte, würde die Wiederverwendbarkeit und generelle Entwicklungsarbeit positiv beeinflussen, beziehungsweise vereinfachen"'~\citep[S. 3]{bericht}, wird von den Autoren des Berichts daher als Bestreben hinter dem COSIMA-Projekt angeführt.
  
  Neben der Notwendigkeit ein \emph{Meta-Framework}\footnote{Das ursprüngliche Ziel war tatsächlich ein Framework zu schaffen. Erst während der Validierung im Rahmen dieser Arbeit ist zu Tage gekommen, dass es sich mehr um eine Architektur handelt und weniger um ein Framework. Im weiteren Verlauf wird darauf jedoch noch genauer eingegangen.} zu schaffen, führen die Autoren als weiteren Beweggrund das Fehlen einer Architektur für Multimediaanwendungen an. Innerhalb dieser Architektur könnten sich Anwendungsentwickler wesentlicher effektiver bewegen und müssten nicht erst eine eigene Architektur von Grund auf entwerfen.
  
  Da sich mit den meisten bestehenden Multimedia-Rahmenwerke keine verteilten Anwendungen realisieren lassen, lag auch dieser Aspekt von Beginn an im Fokus der Konzeptionierung. Als Grundlage eine geeignete Architektur zu konzeptionieren, die es ermöglicht, verteilte Anwendungen zu realisieren, diente das Konzept der \emph{Service-oriented Architecture} (SOA\abk{SOA}{Service-oriented Architecture}) oder \emph{Dienst-orientierten Architektur}.
  
  Die hier aufgeführten Punkte haben initial die Entwicklung eines Rahmenwerkes motiviert, dass später im COSIMA-Projekt aufgehen sollte. Die im Verlauf der Projektarbeit entwickelten Ziele von COSIMA sind im nächsten Abschnitt zusammen gefasst.
  
\subsection{Ziele} % (fold)
\label{sub:ziele}

  Das \emph{Mission Statement} des COSIMA-Projekts fasst alle Ziele des Projekts in einer Kernaussage zusammen:

  \begin{quote}
    \emph{``MIAV ist ein integratives, komponentenbasiertes Meta-Framework mit gezielter Ausrichtung auf Multimediaverarbeitung. Es vereinfacht die Entwicklung von verteilten Multimedia-Applikationen durch eine flexible, dienst-orientierte Architektur. Die Wiederverwendbarkeit von Komponenten und bestehenden Frameworks wird dadurch begünstigt.''} (aus~\citep[S. 2]{bericht})\footnote{Die Bezeichnung "`COSIMA"' hat das Projekt erst nach Fertigstellung des Berichts erhalten, daher findest sich hier noch die zuvor verwandte provisorische Bezeichnung \emph{MIAV-Framework}.}
  \end{quote}

  Neben den zentralen Aspekten \emph{dienst-orientierte Architektur}, \emph{Integration} und \emph{Meta-Framework}, die im Abschnitt zuvor bereits dargestellt wurden, nennen die Autoren hier zusätzlich noch die Aspekte der \emph{komponentenbasierten Architektur}, \emph{Wiederverwendbarkeit} und natürlich der \emph{Medienverarbeitung}.
  
  Neben den hier genannten Zielen, die das COSIMA-Projekt zu erreichen versucht, zeichnet sich das Projekt durch seine spezifischen Charakteristika in Bezug auf andere Multimedia-Rahmenwerke aus. Diese Alleinstellungsmerkmale werden im nächsten Abschnitt genauer betrachtet.

  % - Welche Ziele verfolgt das COSIMA-Projekt?
  % - Warum handelt es sich um eine Architektur und nicht um ein Framework!!!! (Im Bericht noch anders, irgendwie muss das hier verwurstet werden!)
  % - Weiterentwicklung der Definition seit dem Bericht

% subsection ziele (end)
  
% section motivation_und_ziele_von_cosima (end)

\section{Alleinstellungsmerkmale} % (fold)
\label{sec:alleinstellungsmerkmale}

  Aus den in Abschnitt~\ref{sub:ziele} dargestellten Zielen des COSIMA-Projekts lassen sich die folgenden Merkmale extrahieren, die COSIMA im Bereich der Multimedia-Rahmenwerke und -Anwendungen einmalig machen~\citep[S. 3f]{bericht}:
  
  \begin{description}
    \item[Verteiltheit] COSIMA ist konzeptioniert als ein verteiltes System.
    \item[Dienstorientierung] Angelehnt an die \emph{Service-Oriented Architecture} (SOA), sind die Bausteine in COSIMA als Dienste modelliert.
    \item[Integration] Bestehende Frameworks können in Form von Diensten angeboten und so ihre Funktionalität eingebunden werden.
    \item[Erweiterbarkeit] Die Dienstorientierung erlaubt die Einbindung eigener Komponenten.
    % TODO - Die Skalierbarkeit muss hier noch weiter beschrieben werden. Eine reine Verteilung führt noch zu keiner gute Skalierbarkeit eines Systems.
    \item[Skalierbarkeit] Als verteiltes System können Dienste auf verschiedene Systeme ausgelagert werden, es gibt kein monolithisches System\footnote{die Verschiebung des Flaschenhalses von einem System hat zur Folge, dass die Verbindung zwischen den Diensten entsprechend angelegt sein muss. [\textbf{QUELLE}]}.
    \item[Medienobjekt-Modellierung] Modellierung von Medien in ganzheitlicher Betrachtungsweise von Rohdaten und Metadaten in einem Objekt.
    \item[Meta-Ebene] COSIMA fokussiert nicht auf Datensicht oder Metadatensicht sondern abstrahiert auf höhere Ebene.
    \item[Medienverarbeitung] Ganzheitliche Sicht auf Medienverarbeitung: Produktion, Verarbeitung, Transformation, Anreicherung, Wiedergabe, Ausgabe von Daten und Metadaten
    \item[Architektur] COSIMA stellt eine Architektur für Multimediaanwendungen
  \end{description}

% section alleinstellungsmerkmale (end)

\section{Architektur} % (fold)
\label{sec:architektur}

  - Vorstellen der Architektur
  - Kurze Historie der Architektur und Entwicklung bis jetzt!
  - Konzept darstellen (kurz), dabei verweisen auf den Bericht
  
\subsection{zentrale Komponenten} % (fold)
\label{sub:zentrale_komponenten}

  - Kernpunkte der Architektur herausarbeiten
  - Diese Kernpunkte müssen in der Realisierung/Validierung entsprechend besondere Berücksichtigung finden
  
\subsubsection{Kernkomponenten} % (fold)
\label{ssub:kernkomponenten}

   - Zentrale Komponente der Architektur erläutern
   - Quelle-Komponente-Senke Prinzip
   - Begrifflichkeit erläutern: Producer-Transformer-Consumer

% subsubsection kernkomponenten (end)

\subsection{Service Registry} % (fold)
\label{sub:service_registry}

  - Zentrale Stelle zur Registrierung/Auffindung von Services
  - Definition der allgemeinen Schnittstelle für COSIMA-Services

% subsection service_registry (end)

\subsubsection{Service Komposition} % (fold)
\label{ssub:service_komposition}

  - Definitionen von "`Workflow"' und "`Prozess"' in Zusammenhang auf die Entwürfe (vielleicht )
  - BPEL/Orchestrierung/Choreographie mit Quellenangaben erläutern
  - vielleicht macht dieser Abschnitt überhaupt keinen Sinn!
  - Die grundsätzliche Möglichkeit der Verwendung einer Prozessbeschreibungssprache diskutieren

% subsubsection service_komposition (end)

\subsubsection{Nachrichtensystem} % (fold)
\label{ssub:nachrichtensystem}

  - Verwendung
  - Einfluss auf Komposition

% subsubsection nachrichtensystem (end)

\subsubsection{Infrastruktur} % (fold)
\label{ssub:infrastruktur}

  - Beschreibung der (abstrakten) Infrastrukturelemente
  - Eigentlich gehört auch das Medienobjekt grob zur Infrastruktur, ist jedoch so wichtig, dass es eigenen Punkt erhält

\paragraph{Persistenz} % (fold)
\label{par:persistenz}

  - Persistenzschicht
  - jetzt nicht wichtig

% paragraph persistenz (end)

\paragraph{Timing} % (fold)
\label{par:timing}

  - gedacht für Synchronisation
  - auch nicht im Fokus der Betrachtungen

% paragraph timing (end)

% subsubsection infrastruktur (end)

\subsubsection{Medienobjekt} % (fold)
\label{ssub:medienobjekt}

  - Begründung warum separat betrachtet von Infrastruktur
  - Media Broker?!
  - Bedeutung für COSIMA
  - Verantwortlichkeiten

% subsubsection medienobjekt (end)

% subsection zentrale_komponenten (end)

\subsection{Offene Fragen} % (fold)
\label{sub:offene_fragen}

  - Was ist zu diesem Zeitpunkt noch offen?
  - Was kann auch am Ende dieser Arbeit nicht abschließend geklärt sein?
  - Kann hier überhaupt noch von einem "`Framework"' gesprochen werden?

% subsection offene_fragen (end)

% section architektur (end)

% chapter eine_dienstorientierten_multimediaarchitektur (end)
