%!TEX root = /Users/dbreuer/Documents/Work/_FH/_Master/master_thesis/Main/Master Thesis.tex

\chapter{COSIMA - Eine dienstorientierten Multimediaarchitektur} % (fold)
\label{cha:eine_dienstorientierten_multimediaarchitektur}

  Die Grundlage dieser Arbeit ist eine dienstorientierte Multimediaarchitektur, die in diesem Kapitel im Detail vorgestellt werden soll.
  
\section{Idee und Motivation hinter COSIMA} % (fold)
\label{sec:idee_und_motivation_hinter_cosima}

  - Was ist die Idee hinter COSIMA?
  - Was war die Motivation der Entwicklung?
  - Kurze Historie der Architektur und Entwicklung bis jetzt!
  - Konzept darstellen (kurz), dabei verweisen auf den Bericht

% section idee_und_motivation_hinter_cosima (end)

\section{Ziele} % (fold)
\label{sec:ziele}

  - Welche Ziele verfolgt das COSIMA-Projekt?
  - Warum handelt es sich um eine Architektur und nicht um ein Framework!!!! (Im Bericht noch anders, irgendwie muss das hier verwurstet werden!)
  - Weiterentwicklung der Definition seit dem Bericht

% section ziele (end)

\section{Alleinstellungsmerkmale} % (fold)
\label{sec:alleinstellungsmerkmale}

  - Was zeichnet COSIMA aus.
  
  \begin{description}
    \item[Verteiltheit] COSIMA ist konzeptioniert als ein verteiltes System.
    \item[Dienstorientierung] Angelehnt an die \emph{Service-Oriented Architecture} (SOA), sind die Bausteine in COSIMA als Dienste modelliert.
    \item[Integration] Bestehende Frameworks können in Form von Diensten angeboten und so ihre Funktionalität eingebunden werden.
    \item[Erweiterbarkeit] Die Dienstorientierung erlaubt die Einbindung eigener Komponenten.
    % TODO - Die Skalierbarkeit muss hier noch weiter beschrieben werden. Eine reine Verteilung führt noch zu keiner gute Skalierbarkeit eines Systems.
    \item[Skalierbarkeit] Als verteiltes System können Dienste auf verschiedene Systeme ausgelagert werden, es gibt kein monolithisches System\footnote{die Verschiebung des Flaschenhalses von einem System hat zur Folge, dass die Verbindung zwischen den Diensten entsprechend angelegt sein muss. [\textbf{QUELLE}]}.
    \item[Medienobjekt-Modellierung] Modellierung von Medien in ganzheitlicher Betrachtungsweise von Rohdaten und Metadaten in einem Objekt.
    \item[Meta-Ebene] COSIMA fokussiert nicht auf Datensicht oder Metadatensicht sondern abstrahiert auf höhere Ebene.
    \item[Medienverarbeitung] Ganzheitliche Sicht auf Medienverarbeitung: Produktion, Verarbeitung, Transformation, Anreicherung, Wiedergabe, Ausgabe von Daten und Metadaten
    \item[Architektur] COSIMA stellt eine Architektur für Multimediaanwendungen
  \end{description}

% section alleinstellungsmerkmale (end)

\section{Architektur} % (fold)
\label{sec:architektur}

  - Vorstellen der Architektur
  
\subsection{zentrale Komponenten} % (fold)
\label{sub:zentrale_komponenten}

  - Kernpunkte der Architektur herausarbeiten
  - Diese Kernpunkte müssen in der Realisierung/Validierung entsprechend besondere Berücksichtigung finden
  
\subsubsection{Kernkomponenten} % (fold)
\label{ssub:kernkomponenten}

   - Zentrale Komponente der Architektur erläutern
   - Quelle-Komponente-Senke Prinzip
   - Begrifflichkeit erläutern: Producer-Transformer-Consumer

% subsubsection kernkomponenten (end)

\subsection{Service Registry} % (fold)
\label{sub:service_registry}

  - Zentrale Stelle zur Registrierung/Auffindung von Services
  - Definition der allgemeinen Schnittstelle für COSIMA-Services

% subsection service_registry (end)

\subsubsection{Service Komposition} % (fold)
\label{ssub:service_komposition}

  - Definitionen von "`Workflow"' und "`Prozess"' in Zusammenhang auf die Entwürfe (vielleicht )
  - BPEL/Orchestrierung/Choreographie mit Quellenangaben erläutern
  - vielleicht macht dieser Abschnitt überhaupt keinen Sinn!
  - Die grundsätzliche Möglichkeit der Verwendung einer Prozessbeschreibungssprache diskutieren

% subsubsection service_komposition (end)

\subsubsection{Nachrichtensystem} % (fold)
\label{ssub:nachrichtensystem}

  - Verwendung
  - Einfluss auf Komposition

% subsubsection nachrichtensystem (end)

\subsubsection{Infrastruktur} % (fold)
\label{ssub:infrastruktur}

  - Beschreibung der (abstrakten) Infrastrukturelemente
  - Eigentlich gehört auch das Medienobjekt grob zur Infrastruktur, ist jedoch so wichtig, dass es eigenen Punkt erhält

\paragraph{Persistenz} % (fold)
\label{par:persistenz}

  - Persistenzschicht
  - jetzt nicht wichtig

% paragraph persistenz (end)

\paragraph{Timing} % (fold)
\label{par:timing}

  - gedacht für Synchronisation
  - auch nicht im Fokus der Betrachtungen

% paragraph timing (end)

% subsubsection infrastruktur (end)

\subsubsection{Medienobjekt} % (fold)
\label{ssub:medienobjekt}

  - Begründung warum separat betrachtet von Infrastruktur
  - Media Broker?!
  - Bedeutung für COSIMA
  - Verantwortlichkeiten

% subsubsection medienobjekt (end)

% subsection zentrale_komponenten (end)

\subsection{Offene Fragen} % (fold)
\label{sub:offene_fragen}

  - Was ist zu diesem Zeitpunkt noch offen?
  - Was kann auch am Ende dieser Arbeit nicht abschließend geklärt sein?

% subsection offene_fragen (end)

% section architektur (end)

% chapter eine_dienstorientierten_multimediaarchitektur (end)
