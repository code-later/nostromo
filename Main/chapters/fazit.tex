\chapter{Fazit} % (fold)
\label{cha:fazit}

  In dieser Arbeit wurde die Konzeption, prototypische Implementierung sowie die Validierung einer dienstorientierten Architektur für Multimediaanwendungen behandelt. Bei der Konzeption wurde zunächst ersichtlich, dass eine dienstorientierte Architektur eine Vielzahl von unterschiedlichen Komponenten aufweist, von denen jede einzelne bereits ein gewisses Maß an Komplexität aufweist. Unter der Hinterzunahme der zusätzlichen Eigenschaften, die die Integration von Multimedia gefordert hat, entstand eine komplexe Gesamtarchitektur.
  
  Durch diese Komplexität war es notwendig die jeweiligen Komponenten zunächst isoliert zu betrachten und im Anschluss in den größeren Kontext einzuordnen. Durch dieses Vorgehen wurde zum einen ein sehr viel besseres Verständnis über COSIMA entwickelt, zum anderen lassen sich die Komponenten in weiteren Arbeiten dadurch einzeln leichter betrachten.
  
  In seiner Gesamtheit weist das COSIMA-Projekt schon ein sehr hohes Maß an Vollständig\-keit auf. Die wesentlichen Aspekte sind durchweg bekannt und entsprechend diskutiert worden. Lediglich in vertikaler Richtung fehlt es Aspekten, wie der Synchronisation von Medien oder der Behandlung von Metadaten noch an Substanz.
  
  Entsprechend der Konzeption konnte auch die Implementierung in horizontaler Ebene fast vollständig umgesetzt werden. Auch hier wurde für nachfolgende Arbeiten diese Grundlage geschaffen. Dies wurde vor allem auch durch das iterative Entwicklungsvorgehen und durch den Einsatz eines szenariobasierten Ansatzes begünstigt. Durch den vermehrten Einsatz von etablierten Entwurfsmustern und Entwicklungsparadigmen sowie der Verwendung der Programmiersprache Java während der Implementierung wurde auch in diesem Teil eine solide Basis für Folgeprojekte geschaffen.
  
  Es lässt sich also festhalten, dass die beiden aufgestellten Ziele im Rahmen der gestellten Aufgaben für diese Arbeit als erfüllt gelten können. Dennoch müssen einige Punkte kritisch angemerkt werden.
  
  Die Nichtverwendung einer etablierten Prozessbeschreibungssprache stellt eines der Hauptprobleme dar. Zwar erlaubt die gewählte Lösung in dieser Arbeit ein besseres Verständnis über die internen Abläufe bei der Servicekomposition, für eine langfristige Weiterentwicklung sollte diese Komponente jedoch ausgetauscht werden. Stattdessen sollte die Eignung und Anpassung bestehender Lösungen weiterverfolgt werden, wie sie bereits bei \citep{samma08} begonnen wurde.
  
  Ebenfalls nur rudimentär ist die Integration von Synchronisation durchgeführt worden. Es wurde vor allem klar, dass es sich dabei um ein sehr komplexes Themengebiet handelt. In \citep{antons09} findet aber bereits eine intensivere Auseinandersetzung mit der Einbindung von Synchronisation in das COSIMA-Projekt statt. In Zukunft sollten Arbeiten die dort gefundenen Ergebnisse mit der hier vorgestellten Implementierung in Einklang bringen.
  
  Der letzte Punkt, der weitestgehend ausgelassen wurde, ist die Betrachtung von Metadaten. Da dieser aber nicht zur essentiellen Funktionalität des umgesetzten Szenarios gehört, wurde bisher nur eine Schnittstelle vorgesehen. In \citep{lehmann09} findet sich eine dedizierte Betrachtung dieses Themas im Kontext von COSIMA.
  
  Zu den anderen Punkten kann festgehalten werden, dass sie trotz ihres prototypischen Charakters durchaus so implementiert sind, dass sie sich auch für den Einsatz in Folgeprojekten eignen.
  
  Abschließend lässt sich festhalten, dass das gesamte COSIMA-Projekt einen innovativen und breiten Themenbereich im Rahmen der Medieninformatik bietet. Nachdem in dieser Arbeit zum ersten Mal eine funktionierende Implementierung der grundlegenden Architektur geliefert wurde, haben künftige Arbeiten in diesem Bereich eine robuste Basis zur Hand, um in den unterschiedlichen Teilbereichen des Projekts in die Tiefe gehen zu können.

% chapter fazit (end)