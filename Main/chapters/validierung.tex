%!TEX root = /Users/dbreuer/Documents/Work/_FH/_Master/master_thesis/Main/Master Thesis.tex

\chapter{Validierung der Architektur} % (fold)
\label{cha:validierung_der_architektur}

  \begin{definition}[Validierung]\label{def:validierung}
    "`The process of evaluating a system or component during or at the end of the development process to determine whether it satisfies specified requirements. Contrast with: verification"'~\emph{\citep{ieee90sg}}
  \end{definition}
  
  \begin{definition}[Verifikation]\label{def:verifikation}
    "`The process of evaluating a system or component to determine whether the products of a given development phase satisfy the conditions imposed at the start of that phase. Contrast with: validation. (2) Formal proof of program correctness."'~\emph{\citep{ieee90sg}}
  \end{definition}
  
  \begin{definition}[Verifikation und Validierung (V\&V)]\label{def:verifikation_und_validierung_vv}
    "`The process of determining whether the requirements for a system or component are complete and correct, the products of each development phase fulfill the requirements or conditions imposed by the previous phase, and the final system or component complies with specified requirements. See also: independent verification and validation"'~\emph{\citep{ieee90sg}}
  \end{definition}
  
  \begin{definition}[Evaluation]\label{def:evaluation}
    "`[Die] sach- und fachgerechte Bewertung"'\footnote{aus: Duden - Deutsches Universal Wörterbuch A-Z, 3. Aufl., 1996}
  \end{definition}
  
  Im Kontext dieser Arbeit ist fachgerecht wie folgt zu verstehen ...

  Einführung in die Validierung~\citep{balzert1998lst}
  
\section{Definition} % (fold)
\label{sec:definition}

  - Validierung vs Evaluation (ausführlich)

% section definition (end)

\section{Vorgehen} % (fold)
\label{sec:vorgehen}

  - Vorgehen bei der Validierung darstellen
  - Quellen, die dieses Vorgehen beschreiben
  - Vorgehen muss noch gefunden werden (siehe auch Szenarien!)
  
  - Vorgehen beschreiben und Ergebnisse herausarbeiten

  - BPEL/Orchestrierung/Choreographie mit Quellenangaben erläutern. Auswirkungen auf die ursprüngliche Konzeption der Architektur darlegen (Messaging System vs Aufruf vom Workflow System)

% section vorgehen (end)

\section{Ergebnisse} % (fold)
\label{sec:ergebnisse}

  - Was sind die Ergebnisse der Validierung?

% section ergebnisse (end)

\section{Auswirkungen} % (fold)
\label{sec:auswirkungen}

  - Welche Auswirkungen haben die Ergebnisse auf die Architektur?

% section auswirkungen (end)

% chapter validierung_der_architektur (end)