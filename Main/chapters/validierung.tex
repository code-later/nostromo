%!TEX root = /Users/dbreuer/Documents/Work/_FH/_Master/master_thesis/Main/Master Thesis.tex

\chapter{Validierung der Architektur} % (fold)
\label{cha:validierung_der_architektur}

  - Einführung in die Validierung
  
\section{Definition} % (fold)
\label{sec:definition}

  - Validierung vs Evaluation (ausführlich)

% section definition (end)

\section{Vorgehen} % (fold)
\label{sec:vorgehen}

  - Vorgehen bei der Validierung darstellen
  - Quellen, die dieses Vorgehen beschreiben
  - Vorgehen muss noch gefunden werden (siehe auch Szenarien!)
  
  - Vorgehen beschreiben und Ergebnisse herausarbeiten

  - BPEL/Orchestrierung/Choreographie mit Quellenangaben erläutern. Auswirkungen auf die ursprüngliche Konzeption der Architektur darlegen (Messaging System vs Aufruf vom Workflow System)

% section vorgehen (end)

\section{Ergebnisse} % (fold)
\label{sec:ergebnisse}

  - Was sind die Ergebnisse der Validierung?

% section ergebnisse (end)

\section{Auswirkungen} % (fold)
\label{sec:auswirkungen}

  - Welche Auswirkungen haben die Ergebnisse auf die Architektur?

% section auswirkungen (end)

% chapter validierung_der_architektur (end)