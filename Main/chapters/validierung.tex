%!TEX root = /Users/dbreuer/Documents/Work/_FH/_Master/master_thesis/Main/Master Thesis.tex

\chapter{Validierung der Architektur} % (fold)
\label{cha:validierung_der_architektur}

  Bevor eine Validierung der Architektur gegen das Szenario stattfinden kann, wird zunächst definiert, was Validierung ist und wie es im Kontext dieser Arbeit verstanden wird.
  
\section{Definition} % (fold)
\label{sec:definition_validierung}
  
  Eine Definition für Validierung liefert das offizielle Glossar des Software-Engineering der IEEE:

  \begin{definition}[Validierung (IEEE)]\label{def:validierung_ieee}
    "`The process of evaluating a system or component during or at the end of the development process to determine whether it satisfies specified requirements. Contrast with: verification"'~\emph{\citep{ieee90sg}}
  \end{definition}
  
  In dieser Definition taucht der Begriff Evaluation auf, der wie folgt definiert werden kann:
  
  \begin{definition}[Evaluation]\label{def:evaluation}
    "`[Die] sach- und fachgerechte Bewertung"'\footnote{aus: Duden - Deutsches Universal Wörterbuch A-Z, 3. Aufl., 1996}
  \end{definition}
  
  Eine "`fachgerechte Bewertung"' ist für den gegebenen Kontext so zu verstehen, dass die prototypische Implementierung der Architektur anhand bestimmter Kriterien bewertet wird. Wie in Kapitel~\ref{cha:szenario} jedoch bereits verdeutlicht wurde, wird im Rahmen dieser Arbeit keine Evaluierung der Architektur vorgenommen, sondern eine Validierung. Aus diesem Grunde wird hier daher die folgende Definition verwendet:
  
  \begin{definition}[Validierung (Balzert)]\label{def:validierung_balzert}
    "`Unter Validation wird die Eignung bzw. der Wert eines Produktes bezogen auf seinen Einsatzzweck verstanden."'~\emph{\citep[S. 101]{balzert1998lst}}
  \end{definition}
  
  Sie drückt wesentlich deutlicher aus, dass es darum geht, zu überprüfen, ob sich die Architektur grundsätzlich für die Konstruktion von verteilten Multimediaanwendungen eignet.
  
  Der Begriff der \emph{Validierung} steht in der Regel nicht für sich allein, sondern immer gemeinsam mit dem Begriff der \emph{Verifikation}. Auch wenn dieser für diese Arbeit keine weitere Relevanz hat, soll er der Vollständigkeit halber ebenfalls definiert werden:
  
  \begin{definition}[Verifikation (IEEE)]\label{def:verifikation_ieee}
    "`The process of evaluating a system or component to determine whether the products of a given development phase satisfy the conditions imposed at the start of that phase. Contrast with: validation. (2) Formal proof of program correctness."'~\emph{\citep{ieee90sg}}
  \end{definition}
  
  \begin{definition}[Verifikation (Balzert)]\label{def:verifikation_balzert}
    "`Unter Verifikation wird die Überprüfung der Übereinstimmung zwischen einem Software-Produkt und seiner Spezifikation verstanden."'~\emph{\citep[S. 101]{balzert1998lst}}
  \end{definition}
  
  Bei~\citep{boehm1984vv} wird die Bedeutungen von Verifikation und Validierung noch einmal auf den Punkt gebracht:
  
  \begin{description}
    \item[Verifikation] "`Am I building the product right?"'~\citep[S. 75]{boehm1984vv}
    \item[Validierung] "`Am I building the right product?"'~\citep[S. 75]{boehm1984vv}
  \end{description}
  
  Es wird hier auch deutlich, warum zum gegebenen Zeitpunkt nur eine Validierung stattfinden kann und nicht zusätzlich auch eine Verifikation: Es existiert bis dato noch keine formale Spezifikation des Produkts, in diesem Fall der Architektur.
  
  % \begin{definition}[Verifikation und Validierung (V\&V)]\label{def:verifikation_und_validierung_vv}
  %   "`The process of determining whether the requirements for a system or component are complete and correct, the products of each development phase fulfill the requirements or conditions imposed by the previous phase, and the final system or component complies with specified requirements. See also: independent verification and validation"'~\emph{\citep{ieee90sg}}
  % \end{definition}

% section definition (end)

\section{Ergebnisse aus der Santiago Anwendung} % (fold)
\label{sec:ergebnisse_aus_der_santiago_anwendung}

% section ergebnisse_aus_der_santiago_anwendung (end)

\section{Ergebnisse aus der Nerstrand Anwendung} % (fold)
\label{sec:ergebnisse_aus_der_nerstrand_anwendung}

% section ergebnisse_aus_der_nerstrand_anwendung (end)

\section{Probleme und Lösungen} % (fold)
\label{sec:probleme_und_loesungen_architektur}

  - Welche Probleme sind aufgetaucht?
  - Wie wurden diese gelöst?
  - Auswirkungen auf die Architektur!
  - Workflow ohne externes Messaging System möglich!?
  
  Die einzelnen Medienobjekte werden über Medienbroker vermittelt. Eigenes \verb!cosima://!-Protokoll für den Zugriff auf Medienbroker etablieren.
    
  Keine adäquate Fehlerbehandlung.
  
  Keine hochwertige Fehlerrobustheit. Nur rudimentäres Logging.
  
  Verteilte Ausführung der Komposition setzt ebenfalls eine verteilte Hinterlegung des aktuell Fortschritts der Ausführung voraus, Möglichkeit besteht über das \verb!ProcessStore!-Interface.
  
  Keine Möglichkeit zur asynchronen Ausführung von einzelnen Diensten bis jetzt implementiert. Ist nicht immer angemessen abhängig vom Art des Dienstes, doch mit unter notwendig (\textbf{Stichwort}: \emph{Streaming-Dienste}).
  
  Die bisherige Umsetzung der Komposition der einzelnen Dienste ist nur wenig flexibel und eignet sich wahrscheinlich nicht für Realisierung von komplexeren Szenarien. Dafür wäre im jeden Fall die Umsetzung auf Basis eines Zustandsautomaten notwenig, viel eher sollte aber die Verwendung einer etablierten Prozessbeschreibungssprache in Betracht gezogen werden. Dazu sind jedoch weitere Betrachtungen notwendig, wie sie unter anderem in~\citep{samma08} gemacht wurden.

  Bereits bei dieser einfachen Anwendung besteht auch ein Bedarf nach Synchronisation. Die Tranformer-Komponente, die die Diashow und das Musikstück verarbeitet muss dabei beide Medien synchronisieren. Es findet hier allerdings nur eine implizite Synchronisation statt, da sie in keiner Weise spezifiziert wurde. Des Weiteren wird sie auf Tranformer-Seite durchgeführt, was der Synchronisation durch Multiplex-Datenströme nach Steinmetz entspricht~\citep[S. 609]{multimedia_technologie} (siehe auch Abschnitt \ref{par:besonderheiten_bei_verteilten_umgebungen}).
  
  Etablierung eines Logging Service. Zur Zeit nur ein Logging Singleton. Nur bedingt für verteilte Anwendung geeignet, da kein zentrales Log der Gesamtapplikation verfügbar ist.
  
  Dienste, die einmal gestartet immer laufen sollten wieder gestoppt werden können (als Beispiel Streaming).
  
  \verb!MediaStoreFactory! zur dynamischen Ermittlung der optimalen Speicherstrategie für die jeweiligen Mediendaten. Dazu auch Einsatz von Metadaten notwendig. MultiMonster Medienserver hier wieder anbringen.
  
  Nachrichtensystem ist zwar vorhanden, wird zur Zeit jedoch nicht verwendet, da das gewählte Nachrichtensystem auf dem Publish-Subscribe-Verfahren aufsetzt und daher nur für asynchrone Kommunikation sinnvoll einsetzbar ist. Eine Kommunikation mit den einzelnen medienverarbeitenden Diensten von Seiten der Servicekomposition ist daher nicht sinnvoll. Selbst für die Fälle, in denen Dienste asynchron aufgerufen werden müssen, beispielsweise Streamingdienste, lässt sich das ebenso direkt in der Kompositionskomponente umsetzen (Streamingdienste müssen dabei noch nicht mal asynchron aufgerufen werden. Der Aufruf kann auch synchron erfolgen und die Antwort ist in diesem Fall irrelevant.). Das Nachrichtensystem ließe sich aber für die Übermittlung von Synchronisationsspezifikationen verwenden.

% section probleme_und_loesungen_architektur (end)

\section{Vorgehen} % (fold)
\label{sec:vorgehen}

  - Vorgehen bei der Validierung darstellen
  - Quellen, die dieses Vorgehen beschreiben
  - Vorgehen muss noch gefunden werden (siehe auch Szenarien!)
  
  - Vorgehen beschreiben und Ergebnisse herausarbeiten

  - BPEL/Orchestrierung/Choreographie mit Quellenangaben erläutern. Auswirkungen auf die ursprüngliche Konzeption der Architektur darlegen (Messaging System vs Aufruf vom Workflow System)

% section vorgehen (end)

\section{Ergebnisse} % (fold)
\label{sec:ergebnisse}

  - Was sind die Ergebnisse der Validierung?

% section ergebnisse (end)

\section{Auswirkungen} % (fold)
\label{sec:auswirkungen}

  - Welche Auswirkungen haben die Ergebnisse auf die Architektur?

% section auswirkungen (end)

% chapter validierung_der_architektur (end)