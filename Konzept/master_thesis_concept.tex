%! program = pdflatex

%
%  Konzept zur Master Thesis von Dirk Breuer
%
%  Created by Dirk Breuer on 2008-05-22.
%  Copyright (c) 2008 Dirk Breuer. All rights reserved.
%

%%% BEGIN OF PREAMBEL

\documentclass[11pt,headsepline,a4paper,bibtotoc,liststotoc,DIV12,BCOR12mm]{scrartcl}

% Use utf-8 encoding for foreign characters
\usepackage[utf8]{inputenc}
\usepackage[ngerman]{babel}

\usepackage{setspace}
% \onehalfspacing
\raggedbottom

% Package for Multicolumn output
\usepackage{multicol}

% Setup for fullpage use
\usepackage{fullpage}

\usepackage{natbib}

% Uncomment some of the following if you use the features
%
% Running Headers and footers
% \usepackage{fancyheadings}

% Multipart figures
%\usepackage{subfigure}

% More symbols
%\usepackage{amsmath}
%\usepackage{amssymb}
%\usepackage{latexsym}

% Surround parts of graphics with box
\usepackage{boxedminipage}

% Package for including code in the document
\usepackage{listings}

% If you want to generate a toc for each chapter (use with book)
\usepackage{minitoc}

% This is now the recommended way for checking for PDFLaTeX:
\usepackage{ifpdf}

%\newif\ifpdf
%\ifx\pdfoutput\undefined
%\pdffalse % we are not running PDFLaTeX
%\else
%\pdfoutput=1 % we are running PDFLaTeX
%\pdftrue
%\fi

\ifpdf
\usepackage[pdftex]{graphicx}
\else
\usepackage{graphicx}
\fi

\pdfpagewidth=\paperwidth
\pdfpageheight=\paperheight
\usepackage[pdftex,plainpages=false,pdfpagelabels,
            pdftitle={Evaluation der MIAV Architektur anhand eines Anwendungszenarios durch prototypische Implementierung},
            pdfauthor={Dirk Breuer - University of Applied Science, Cologne},
            pdfsubject={Konzept zur Master Thesis},
            pdfkeywords={Master Thesis}]{hyperref} % Has to stand at the end of the preamble
%%% END OF PREAMBEL

%%% BEGIN OF TITLE

\titlehead{
	{\large Fachhochschule Köln --- Campus Gummersbach\\
		\hfill }
	Steinmüllerallee~1\\
	51643 Gummersbach
	}
  \subject{Konzept zur Master Thesis}
	\title{Evaluation der MIAV Architektur anhand eines Anwendungszenarios durch prototypische Implementierung (Arbeitstitel)}
	\author{Dirk Breuer\\Medieninformatik Master\\11038920}
	\publishers{
		\center
		\vspace{0.5cm}
		\large University of Applied Science of Cologne\\\textbf{Dozent:} Prof. Dr. Kristian Fischer\\
		\vspace{0.5cm}
    \large \href{mailto:dirk.breuer@gmail.com}{\emph{dirk.breuer@gmail.com}}
		}
	\date{\today}

%%% END OF TITLE

\begin{document}
  
\dosecttoc

\ifpdf
\DeclareGraphicsExtensions{.pdf, .jpg, .tif}
\else
\DeclareGraphicsExtensions{.eps, .jpg}
\fi

\maketitle

\begin{abstract}
  \begin{center}
    \textbf{Abstract\\}
    \vspace{.3cm}
  \end{center}
  \emph{In diesem Konzept wird die zentrale Zielformulierung sowie Problemstellung präsentieren, mit der sich meine Master Thesis beschäftigen wird. Neben diesen zentralen Punkten soll auch das grundsätzliche Vorgehen skizziert werden, ebenso wie ein erster Hinweis auf die Literatur, die verwendet werden soll.}
\end{abstract}
  
% \begin{multicols}{2}

% \newpage

% \newpage
% \tableofcontents
% \newpage

\section{Einleitung} % (fold)
\label{sec:einleitung}

  Im Sommersemester 2007 wurde im Rahmen des \emph{WPF-A Modellierung in Audiovisuellen Medien} ein Projekt gestartet, dessen Ziel es war, ein Multimediaframework zu schaffen, dass sich durch bestimmte Charakteristika von anderen Frameworks abgrenzt und die Betrachtung von Medien aus unterschiedlichen Perspektiven vereint. Im Wintersemester 2007/08 wurde diese Vision in einer Projektarbeit weiter vorangetrieben und die Ziele wurden expliziter formuliert:
  
  \begin{quote}
    \emph{"`MIAV ist ein integratives, komponentenbasiertes Meta-Framework mit gezielter Ausrichtung auf Multimediaverarbeitung. Es vereinfacht die Entwicklung von verteilten Multimedia-Applikationen durch eine flexible, dienst-orientierte Architektur. Die Wiederverwendbarkeit von Komponenten und bestehenden Frameworks wird dadurch begünstigt."'}\footnote{aus dem Institutsbericht.}
  \end{quote}
  
  Am Ende dieser Projektarbeit waren immer noch viele Punkte offen und viele Fragen ungeklärt, was reichlich Raum für Folgearbeiten bietet. Ein wichtiger Punkt der nicht mehr im Rahmen der Projektarbeit behandelt werden konnte, ist die Evaluation der bis dato entstanden MIAV Architektur anhand eines umfangreicheren Szenarios durch die prototypische Implementierung wesentlicher Kernkomponenten der MIAV Architektur. Und eben genau dort soll die Master Thesis ansetzen. Dieses Konzept soll zum besseren Verständnis beitragen, wie genau sich die Master Thesis diesen Punkte behandeln wird und welche Ziele gesteckt werden, die es zu erreichen gilt. Auch ein ersten Hinweis auf die Literatur, die verwendet werden kann, soll gegeben werden.

% section einleitung (end)

\section{Problemstellung} % (fold)
\label{sec:problemstellung}

  Im bisherigen Projektverlauf ist eine Architektur entworfen worden, die die Anforderungen erfüllen soll, die sich aus der Zieldefinition des MIAV-Frameworks ergeben. Die erstellte Architektur ist bisher jedoch nicht methodisch evaluiert worden. Für die weitere Entwicklung einer Architektur für ein so innovatives Framework, dass sich in vielen Bereichen auf Neuland bewegt, ist das jedoch zwingend notwendig.
  
  Zu Beginn des Projektes wurde bereits festgestellt, dass sich eine szenariobasiertes Vorgehen zur Entwicklung eines solchen Frameworks am ehesten anbietet. Daher soll auch weiterhin so verfahren werden und die Architektur auf Grundlage eines Szenarios evaluiert werden. Des Weiteren ist eine prototypische Implementierung zumindest der wesentlichen Aspekte der Architektur unabdingbar für eine aussagekräftige Evaluation. Für die Master Thesis ergeben sich daher zwei zentrale Problemstellungen, die es zu bearbeiten gilt:
  
  \begin{itemize}
    \item Die methodische Entwicklung eines geeigneten Szenarios, dass als Akteur den Anwendungsentwickler hat, der wiederum eine Applikation mit Hilfe des MIAV-Frameworks entwickeln muss. Das Szenario muss dabei so gewählt, dass die Fachdomäne, in der die Applikation entwickelt werden soll, hinreichend bekannt ist.
    \item Die Entwicklung eines Prototypen, der die bisher modellierten Eigenschaften der Architektur funktional umsetzt und sich dazu eignet, gegen das Szenarion evaluiert zu werden. Non-Funktionale Anforderungen können hier zweitrangig betrachtet werden. Unter Umständen ist hier auch eine weitere Modellierung der Architektur im Vorfeld notwendig bevor mit der Implementierung begonnen werden kann.
  \end{itemize}

  Wie eingangs bereits erwähnt, soll die bisherige Architektur anhand des noch zu entwickelnden Szenarios \emph{evaluiert} werden. Neben den beiden zentralen Problemstellungen müssen noch folgende Punkte bezüglich der Evaluation betrachtet werden:

  \begin{itemize}
    \item Der Evaluation selbst muss ein methodische Vorgehen zu Grunde liegen, dass Rücksicht auf die Verwendung von Szenarien nimmt.
    \item Es müssen Kriterien festgelegt werden, anhand derer eine Bewertung stattfinden kann. Diese Kriterien werden wahrscheinlich sehr unterschiedlich ausfallen und sowohl qualitativer wie quantitativer Charakteristik sein. Im Vorfeld sollten daher in jedem Fall solche Kriterien gefunden werden von denen die meiste Aussagekraft erwartet wird und die am ehesten zielfördernd sind.
  \end{itemize}

% section problemstellung (end)

\section{Zieldefinition} % (fold)
\label{sec:zieldefinition}

  Ziel der Evaluation der MIAV-Architektur soll es sein, eine grundsätzliche Aussage treffen zu können, inwieweit die Ziele\footnote{Hierunter fällt auch in wie weit sich die Alleinstellungsmerkmale des Frameworks bereits ausgeprägt sind.} des MIAV Projekts bereits erreicht und welche noch nicht erreicht wurden. Im Einzelnen sind diese Ziele:
  
  \begin{description}
  	\item[Verteiltheit] MIAV ist konzeptioniert als ein verteiltes System.
  	\item[Dienstorientierung] Angelehnt an die \emph{Service-Oriented Architecture}  (SOA), sind die Bausteine in MIAV als Dienste modelliert.
  	\item[Integration] Bestehende Frameworks können in Form von Diensten angeboten und so ihre Funktionalität eingebunden werden.
  	\item[Erweiterbarkeit] Die Dienstorientierung erlaubt die Einbindung eigener Komponenten.
  	% TODO - Die Skalierbarkeit muss hier noch weiter beschrieben werden. Eine reine Verteilung führt noch zu keiner gute Skalierbarkeit eines Systems.
  	\item[Skalierbarkeit] Als verteiltes System können Dienste auf verschiedene Systeme ausgelagert werden, es gibt kein monolithisches System\footnote{die Verschiebung des Flaschenhalses von einem System hat zur Folge, dass die Verbindung zwischen den Diensten entsprechend angelegt sein muss.}.
  	\item[Medienobjekt-Modellierung] Modellierung von Medien in ganzheitlicher Betrachtungsweise von Rohdaten und Metadaten in einem Objekt.
  	\item[Meta-Ebene] MIAV fokussiert nicht auf Datensicht oder Metadatensicht sondern abstrahiert auf höhere Ebene.
  	\item[Medienverarbeitung] Ganzheitliche Sicht auf Medienverarbeitung: Produktion, Verarbeitung, Transformation, Anreicherung, Wiedergabe, Ausgabe von Daten und Metadaten
  	\item[Architektur] MIAV stellt eine Architektur für Multimediaanwendungen
  \end{description}
  
  Aber nicht nur soll ein Maß gefunden werden, zu welchem Grad die genannten Ziele und Alleinstellungsmerkmale bereits vorhanden sind, auch soll eine Bewertung abgegeben werden, ob die Architektur des MIAV-Frameworks grundsätzlich "`funktioniert"'. Diese Bewertung kann dann selbst wieder Grundlage für weiter Projekte sein.
  
  Es ist davon auszugehen, dass noch nicht alle Ziele erreicht wurden und die Architektur grundsätzlich zwar funktioniert, jedoch sicher noch Fehler aufweisen wird. Daher sollte ein weiteres Ziel der Master Thesis auch sein konkrete Verbesserungsvorschläge zu geben, um die Fehler zu beseitigen und die Zielerreichung voranzutreiben.
  
% section zieldefinition (end)

\section{Vorgehen} % (fold)
\label{sec:vorgehen}

  Nachdem die Problemstellung sowie die Zieldefinitionen geklärt wurden, sollen im folgenden die konkreten Schritte präsentiert werden, um den Problemen zu begegnen und die Ziele zu erreichen.

\paragraph{Einführung und Motivation} % (fold)
\label{par:einfuehrung}
  
  Zu Beginn der Arbeit soll das Thema vorgestellt und kurz in das MIAV Projekt eingeführt werden. Die Motivation einer Evaluation der Architektur soll deutlich werden und warum ein szenariobasierter Ansatz als besonders vorteilhaft bewertet wurde. Die Zieldefinition der Arbeit wird präsentiert und der Rahmen in dem sich die Arbeit bewegen soll abgesteckt. Natürlich wird hier auch der Aufbau der Arbeit vorgestellt.
  
% paragraph einfuehrung (end)

\paragraph{Szenario und Evaluationskriterien} % (fold)
\label{par:szenario_und_evaluationskriterien}

  Nach der Einführung soll das Szenario erarbeitet werden, dass im weiteren Verlauf der Arbeit die Grundlage des Prototyps und der Evaluation sein wird. Aus diesem Grunde sollen an dieser Stelle auch die Kriterien festgelegt werden, anhand derer die Evaluation stattfinden wird. Ebenso soll das Vorgehen bei der Evaluation vorgestellt und begründet werden.
  
  Das Szenario selbst sollte nicht zu umfangreich sein und wesentliche Kernaspekte des Frameworks beleuchten\footnote{Diese Kernaspekte müssen dabei dann definiert werden und es sollte herausgestellt werden, warum es sich um Kernaspekte handelt. Es sollte hier beispielsweise der Aspekt der Synchronisation außen vorgelassen werden.}. Dabei sollte es in gewissem Maße visionär angelegt sein, jedoch nicht zu weit weg. Im Vordergrund steht bei der Arbeit die Evaluation der Architektur. Die Entwicklung des Szenarios ist dazu ein notwendiges Hilfsmittel und sollte demnach nicht zu sehr an Gewicht erhalten. Bei einem Szenario, was allzu fern ab des umsetzbaren ist, kann der Wahrheitsgehalt des Ergebnis der Evaluation zu gering ausfallen. Bisher ist angedacht ein Thema aus dem Bereich der Hochschule als Grundlage für das Szenario zu wählen. Der Vorteil dabei ist, dass die Fachdomäne bekannt, was Eingangs bereits als ein Kriterium für das Szenario genannt wurde.
  
  Ganz konkret könnte sich das Szenario mit einer Verteilung einer Lehrveranstaltung zwischen der Niederlassung der Fachhochschule Gummersbach und Deutz beschäftigen. Wichtig ist hierbei, dass nicht Aspekte aus den Bereichen CSCW/CSCL oder Lerntheorien zu stark integriert werden. Bei der Entwicklung des Szenario ist dies sicherlich einer der schwierigsten Punkte.

% paragraph szenario_und_evaluationskriterien (end)

\paragraph{Architektur und Prototyp} % (fold)
\label{par:architektur_und_prototyp}

  Im weiteren Schritt soll die Architektur wie sie bis dato existiert noch einmal genauer vorgestellt werden. Hier sollen auch die Teilbereiche aufgezeigt werden, die bei der prototypischen Implementierung von besonderem Interesse sein werden. Diese ergeben sich zum Großteil aus dem Szenario, könnten aber auch durch Erkenntnisse aus dem MIAV-Projekt motiviert sein.
  
  Die Entwicklung des Prototypen selbst wird sich dabei in zwei Bereiche unterteilen lassen. Auf der einen Seite muss das Framework prototypisch implementiert werden, zum anderen muss mit diesem Framework das Szenario umgesetzt werden. Schon im Vorfeld kann gesagt werden, dass es sich hierbei um eine umfangreiche Aufgabe handeln wird. Daher sollten Möglichkeiten untersucht werden, den Prototypen in einer Programmiersprache zu implementieren, die von sich aus prototypisches Arbeiten begünstigt. Da der Fokus der Evaluation auf den funktionalen Anforderungen liegen soll und die non-funktionalen nur an zweiter Stelle betrachtet werden, kann sich eine solche Sprache auch nur marginal auf die Ergebnisse der Architektur auswirken. Ein weiteres Argument wäre, dass es sich um eine Architektur handelt, die nach SOA Konzepten aufgebaut ist, wodurch eine Unabhängigkeit der Sprache zusätzlich begünstigt wird. Gegen ein solches Vorgehen, spricht, dass so bestehende Frameworks zur Medienverarbeitung aus dem Java Umfeld (wie etwa JMF) schwerer integriert werden können.
  
  Während der Prototyp entwickelt wird, soll auch ein Protokoll erstellt werden, dass bei der Evaluation und deren Auswertung verwendet werden soll.
  
% paragraph architektur_und_prototyp (end)

\paragraph{Evaluation und Bewertung} % (fold)
\label{par:evaluation_und_bewertung}

  Nachdem der Prototyp implementiert wurde muss eine Evaluierung gegen das Szenario stattfinden. Diese muss nach den Eingangs definierten Kriterien stattfinden. Zur Evaluation wird das Protokoll aus der Prototypentwicklung verwendet. Bei der Evaluation wird ebenfalls ein Protokoll erzeugt, das die Grundlage für die Bewertung der Architektur und des MIAV-Frameworks sein soll. Aus den Bewertung werden schließlich die Handlungsempfehlungen für die weitere Entwicklung des Frameworks abgeleitet. Stärken und Schwachstellen der Architektur sollen klar benannt werden.

% paragraph evaluation_und_bewertung (end)

\paragraph{Fazit und Ausblick} % (fold)
\label{par:fazit_und_ausblick}

  Am Ende der Arbeit soll ein Fazit gezogen werden, allerdings soll es ein Fazit auf mehreren Ebenen geben. So soll nicht nur eine Aussage über die Evaluation gegeben werden, sondern auch wie das Vorgehen an sich zu bewerten ist. Schwachstellen bei der Durchführung des szenariobasierten Ansatz sollten benannt werden und Vorschläge zur Verbesserung aufgezeigt. Auch soll ein (sicher mehr subjektive) Aussage über den Aufwand der Implementierung abgeben werden, sowohl der Implementierung des Frameworks als auch der Implementierung einer Anwendung mit dem Framework.
  
  Die Arbeit wird mit einem generellen Ausblick des MIAV-Projekts enden und wie sich die gewonnen Erkenntnisse weiter verwenden lassen.

% paragraph fazit_und_ausblick (end)

% section vorgehen (end)

\section{Literatur} % (fold)
\label{sec:literatur}

% section literatur (end)

% \newpage
% \bibliographystyle{natdin}
% \bibliographystyle{url}
% \nocite{*}
% \bibliography{bibliography}

% \end{multicols}

\end{document}
