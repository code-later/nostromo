%! program = pdflatex

%
%  Konzept zur Master Thesis von Dirk Breuer
%
%  Created by Dirk Breuer on 2008-05-22.
%  Copyright (c) 2008 Dirk Breuer. All rights reserved.
%

%%% BEGIN OF PREAMBEL

\documentclass[12pt,headsepline,a4paper,bibtotoc,liststotoc,DIV12,BCOR12mm]{scrartcl}

% Use utf-8 encoding for foreign characters
\usepackage[utf8]{inputenc}
\usepackage[ngerman]{babel}

\usepackage{setspace}
\onehalfspacing
\raggedbottom

% Package for Multicolumn output
\usepackage{multicol}

% Setup for fullpage use
\usepackage{fullpage}

\usepackage{natbib}

% Uncomment some of the following if you use the features
%
% Running Headers and footers
% \usepackage{fancyheadings}

% Multipart figures
%\usepackage{subfigure}

% More symbols
%\usepackage{amsmath}
%\usepackage{amssymb}
%\usepackage{latexsym}

% Surround parts of graphics with box
\usepackage{boxedminipage}

% Package for including code in the document
\usepackage{listings}

% If you want to generate a toc for each chapter (use with book)
\usepackage{minitoc}

% This is now the recommended way for checking for PDFLaTeX:
\usepackage{ifpdf}

%\newif\ifpdf
%\ifx\pdfoutput\undefined
%\pdffalse % we are not running PDFLaTeX
%\else
%\pdfoutput=1 % we are running PDFLaTeX
%\pdftrue
%\fi

\ifpdf
\usepackage[pdftex]{graphicx}
\else
\usepackage{graphicx}
\fi

\pdfpagewidth=\paperwidth
\pdfpageheight=\paperheight
\usepackage[pdftex,plainpages=false,pdfpagelabels,
            pdftitle={Konzept zur Master Thesis von Dirk Breuer},
            pdfauthor={Dirk Breuer - University of Applied Science, Cologne},
            pdfsubject={},
            pdfkeywords={Master Thesis}]{hyperref} % Has to stand at the end of the preamble
%%% END OF PREAMBEL

%%% BEGIN OF TITLE

\titlehead{
	{\large Fachhochschule Köln --- Campus Gummersbach\\
		\hfill }
	Steinmüllerallee~1\\
	51643 Gummersbach
	}
	\subject{Advanced Seminar}
	\title{Die Bedeutung von Kontext und dessen Modellierung in adaptiven Hypermedia Systemen}
	\author{Dirk Breuer\\Medieninformatik Master\\11038920}
	\publishers{
		\center
		\vspace{0.5cm}
		\large University of Applied Science of Cologne\\\textbf{Dozent:} Prof. Dr. Kristian Fischer\\
		\vspace{0.5cm}
    \large \href{mailto:dirk.breuer@gmail.com}{\emph{dirk.breuer@gmail.com}}
		}
	\date{\today}

%%% END OF TITLE

\begin{document}
  
\dosecttoc

\ifpdf
\DeclareGraphicsExtensions{.pdf, .jpg, .tif}
\else
\DeclareGraphicsExtensions{.eps, .jpg}
\fi

\maketitle

% \begin{multicols}{2}

\newpage

\begin{abstract}
  \begin{center}
    \textbf{Abstract\\}
    \vspace{.3cm}
  \end{center}
  In diesem Konzept wird die zentrale Zielformulierung sowie Problemstellung präsentieren, mit der sich meine Master Thesis beschäftigen wird. Neben diesen zentralen Punkten soll auch das grundsätzliche Vorgehen skizziert werden, ebenso wie ein erster Hinweis auf die Literatur, die verwendet werden soll.
  
\end{abstract}

\newpage
\tableofcontents
\newpage

\section{Einleitung} % (fold)
\label{sec:einleitung}

  Im Sommersemester 2007 wurde im Rahmen des \emph{WPF-A Modellierung in Audiovisuellen Medien} ein Projekt gestartet, dessen Ziel es war, ein Multimediaframework zu schaffen, dass sich durch bestimmte Charakteristika von anderen Frameworks abgrenzt und die Betrachtung von Medien aus unterschiedlichen Perspektiven vereint. Im Wintersemester 2007/08 wurde diese Vision in einer Projektarbeit weiter vorangetrieben und die Ziele wurden expliziter formuliert:
  
  \begin{quote}
    \emph{"`MIAV ist ein integratives, komponentenbasiertes Meta-Framework mit gezielter Ausrichtung auf Multimediaverarbeitung. Es vereinfacht die Entwicklung von verteilten Multimedia-Applikationen durch eine flexible, dienst-orientierte Architektur. Die Wiederverwendbarkeit von Komponenten und bestehenden Frameworks wird dadurch begünstigt."'}\footnote{aus dem Institutsbericht.}
  \end{quote}
  
  Am Ende dieser Projektarbeit waren immer noch viele Punkte offen und viele Fragen ungeklärt, was reichlich Raum für Folgearbeiten bietet. Ein wichtiger Punkt der nicht mehr im Rahmen der Projektarbeit behandelt werden konnte, ist die Evaluation der bis dato entstanden MIAV Architektur anhand eines umfangreicheren Szenarios durch die prototypische Implementierung wesentlicher Kernkomponenten der MIAV Architektur. Und eben genau dort soll die Master Thesis ansetzen. Dieses Konzept soll zum besseren Verständnis beitragen, wie genau sich die Master Thesis diesen Punkte behandeln wird und welche Ziele gesteckt werden, die es zu erreichen gilt. Auch ein ersten Hinweis auf die Literatur, die verwendet werden kann, soll gegeben werden.

% section einleitung (end)

\section{Problemstellung} % (fold)
\label{sec:problemstellung}

  

% section problemstellung (end)

\section{Zieldefinition} % (fold)
\label{sec:zieldefinition}

% section zieldefinition (end)

\section{Vorgehen} % (fold)
\label{sec:vorgehen}

% section vorgehen (end)

\section{Literatur} % (fold)
\label{sec:literatur}

% section literatur (end)

\newpage
\bibliographystyle{natdin}
% \bibliographystyle{url}
% \nocite{*}
\bibliography{bibliography}

% \end{multicols}

\end{document}
